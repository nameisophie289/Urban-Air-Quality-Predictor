\chapter{Overview}

\section{Background}

Air quality monitoring and forecasting are critical for public health, urban planning, and environmental policy. PM2.5 (particulate matter with diameter $\leq$ 2.5 micrometers) is a key indicator of air pollution, capable of penetrating deep into the respiratory system and causing serious health effects.

\section{Project Description}

This project develops an LSTM-based neural network for hourly PM2.5 (fine particulate matter) prediction in the Sydney urban area. The system ingests six months of air quality measurements and meteorological data, processes and validates the data pipeline, and trains a deep learning model for short-term air quality forecasting.

\section{Objectives}

The main objectives of this project include:
\begin{itemize}
    \item \textbf{Data Pipeline Development}: Create an automated system to collect air quality and weather data from multiple APIs
    \item \textbf{Data Quality Assurance}: Implement validation and preprocessing pipelines for reliable model training
    \item \textbf{Predictive Modeling}: Develop an LSTM neural network capable of forecasting hourly PM2.5 concentrations
    \item \textbf{Operational Forecasting}: Generate 24-hour ahead predictions for practical applications
\end{itemize}

\section{Report Structure}

The structure of this report is organized as follows:
\begin{itemize}
    \item \textbf{Chapter 1:} Overview - Project background, objectives, and scope
    \item \textbf{Chapter 2:} Data Collection - Data sources, collection pipeline, and quality assessment
    \item \textbf{Chapter 3:} Model Architecture - LSTM network design and training configuration
    \item \textbf{Chapter 4:} Results and Conclusion - Performance evaluation and future work
\end{itemize}
