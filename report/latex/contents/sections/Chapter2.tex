\chapter{Data Collection}

\section{Data Sources}

\subsection{OpenAQ API v3 - Air Quality Data}

OpenAQ provides open-access air quality data from government-operated monitoring stations worldwide. Table \ref{tab:pollutants} shows the collected parameters.

\begin{table}[H]
\centering
\caption{Collected Air Quality Parameters from OpenAQ}
\label{tab:pollutants}
\begin{tabular}{llll}
\toprule
\textbf{Parameter} & \textbf{Units} & \textbf{Purpose} & \textbf{Sensor ID} \\
\midrule
PM2.5 & $\mu$g/m$^3$ & Target variable & 25196 \\
PM10 & $\mu$g/m$^3$ & Strong predictor & 25195 \\
NO$_2$ & ppm & Traffic pollution & 25192 \\
SO$_2$ & ppm & Industrial source & 25194 \\
CO & ppm & Combustion indicator & 23019 \\
O$_3$ & ppm & Atmospheric chemistry & 25193 \\
\bottomrule
\end{tabular}
\end{table}

\subsection{Open-Meteo API - Weather Data}

Open-Meteo provides free historical and forecast weather data without API key requirements. Table \ref{tab:weather} shows the collected weather variables.

\begin{table}[H]
\centering
\caption{Collected Weather Variables from Open-Meteo}
\label{tab:weather}
\begin{tabular}{ll}
\toprule
\textbf{Category} & \textbf{Variables} \\
\midrule
Temperature & temperature\_2m, dew\_point\_2m, apparent\_temperature \\
Humidity & relative\_humidity\_2m \\
Precipitation & precipitation, rain \\
Pressure & pressure\_msl, surface\_pressure \\
Wind & wind\_speed\_10m, wind\_direction\_10m, wind\_gusts\_10m \\
Other & cloud\_cover, is\_day, sunshine\_duration \\
\bottomrule
\end{tabular}
\end{table}

\section{Data Quality Summary}

\subsection{Pollutant Data}

\begin{table}[H]
\centering
\caption{Pollutant Data Quality Metrics}
\label{tab:pollutant_quality}
\begin{tabular}{ll}
\toprule
\textbf{Metric} & \textbf{Value} \\
\midrule
Total Records & 4,938 \\
Columns & 7 \\
Duplicate Timestamps & 611 \\
Hourly Gaps & 638 \\
\bottomrule
\end{tabular}
\end{table}

Missing value rates: PM2.5 (2.88\%), PM10 (0.77\%), NO$_2$ (2.47\%), SO$_2$ (2.55\%), CO (3.24\%), O$_3$ (1.96\%).

\subsection{Weather Data}

The weather data contained 4,416 records with 15 columns, zero duplicates, zero hourly gaps, and no missing values.

\subsection{Merged Dataset}

After merging pollutant and weather data:
\begin{itemize}
    \item \textbf{Final Records}: 4,931
    \item \textbf{Total Features}: 21
    \item \textbf{Date Range}: 2025-06-01 07:00 UTC to 2025-12-01 23:00 UTC
\end{itemize}

\subsection{Missing Data Handling}

\begin{enumerate}
    \item \textbf{Linear Interpolation}: Applied on time index for small gaps ($<$ 3.3\%)
    \item \textbf{Forward/Backward Fill}: Used for edge cases
    \item \textbf{Result}: Zero null values in processed dataset
\end{enumerate}

\section{PM2.5 Time Series Visualization}

Figure \ref{fig:timeseries} displays the hourly PM2.5 values over the last month of the study period. The chart reveals daily cyclic patterns typical of urban air pollution---concentrations tend to rise during morning and evening rush hours due to increased traffic and fall overnight. The visible day-to-day variability reflects weather influences (wind dispersing pollutants, rain washing particles from the air) and weekly patterns (lower pollution on weekends).

\begin{figure}[H]
    \centering
    \includegraphics[width=0.95\textwidth]{Images/pm25_timeseries_last_month.png}
    \caption{PM2.5 concentration over the last month of the study period.}
    \label{fig:timeseries}
\end{figure}

\section{Feature Correlations}

Figure \ref{fig:correlation} shows the correlation matrix for all 21 features. Key observations:
\begin{itemize}
    \item PM10 has the strongest correlation with PM2.5 (0.87), which is expected since both measure particulate matter from similar sources
    \item CO, NO$_2$, and SO$_2$ show moderate positive correlations with PM2.5, indicating shared emission sources (vehicles, industry)
    \item Wind speed shows negative correlation---higher winds disperse pollutants, reducing PM2.5 concentrations
\end{itemize}

\begin{figure}[H]
    \centering
    \includegraphics[width=0.85\textwidth]{Images/correlation_heatmap.png}
    \caption{Correlation matrix showing relationships between all 21 features.}
    \label{fig:correlation}
\end{figure}
